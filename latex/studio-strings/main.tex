\documentclass{article}

% Per-assignment macros
\def\studionumber{8}
\def\studiotitle{Strings}

% Imports
\usepackage{graphicx} % Required for inserting images
\usepackage[colorlinks=true, linkcolor=blue, urlcolor=blue, citecolor=blue, anchorcolor=blue]{hyperref}
\usepackage{hhline}
\usepackage{amsmath}

% Titling
\usepackage{titling}
\preauthor{\begin{center}}
\postauthor{\par\end{center}\vspace{-30pt}}
\setlength{\droptitle}{-50pt}

% Geometry

\usepackage{geometry}
\geometry{letterpaper, portrait, margin=1in}

\usepackage[skip=5pt]{parskip}
\newlength\tindent
\setlength{\tindent}{\parindent}
\setlength{\parindent}{0pt}
\renewcommand{\indent}{\hspace*{\tindent}}

% Assignment titling (number, due date, etc)
\title{
    Studio \studionumber: \studiotitle
}
\author{ENGR 103, Winter 2024}
\date{}

% Box environments
\usepackage{tcolorbox}
\usepackage{fancyvrb}
\newenvironment{terminalcommand}
    {\VerbatimEnvironment
    \begin{tcolorbox}[title=Terminal Command,colframe=gray!80!blue,colback=black!80!blue]
    \begin{Verbatim}[formatcom=\color{white}]}
    {\end{Verbatim}
    \end{tcolorbox}}
\newenvironment{terminaloutput}
    {\VerbatimEnvironment
    \begin{tcolorbox}[title=Terminal Output,colframe=gray!80!red,colback=black!80!blue]
    \begin{Verbatim}[formatcom=\color{white}]}
    {\end{Verbatim}
    \end{tcolorbox}}

\newenvironment{hint}
    {\begin{tcolorbox}[title=Hint,colframe=white!70!blue,colback=white]}
    {\end{tcolorbox}}

\newenvironment{sourcecode}[1]
    {\VerbatimEnvironment
    \begin{tcolorbox}[title=\texttt{#1},colframe=gray!80!green,colback=black!80!blue]
    \begin{Verbatim}[formatcom=\color{white}]}
    {\end{Verbatim}
    \end{tcolorbox}}

\newenvironment{tip}
    {\begin{tcolorbox}[title=Tip,colframe=white!70!blue,colback=white]}
    {\end{tcolorbox}}

\newcounter{examplerun}
\newenvironment{examplerun}
    {\begin{tcolorbox}[title=Example Run \refstepcounter{examplerun}\theexamplerun,colframe=black!50!green,colback=white,subtitle style={boxrule=0.4pt,
colback=lightgray!80!green}]}
    {\end{tcolorbox}}
\newcommand{\exampleruninputs}{\tcbsubtitle{Inputs}}
\newcommand{\examplerunoutputs}{\tcbsubtitle{Outputs}}

\newcounter{exampleproblem}
\newcounter{exampleproblemsolution}
\newenvironment{exampleproblem}
    {\setcounter{exampleproblemsolution}{0}\begin{tcolorbox}[title=Example Problem \refstepcounter{exampleproblem}\theexampleproblem,colframe=black!50!green,colback=white,subtitle style={boxrule=0.4pt,
colback=lightgray!80!green}]}
    {\end{tcolorbox}}
\newcommand{\exampleproblemstatement}{\tcbsubtitle{Problem statement}}
\newcommand{\exampleproblemsolution}{\refstepcounter{exampleproblemsolution}\tcbsubtitle{Solution \theexampleproblemsolution}}

\newcommand{\imagewithdefaults}[1]{\includegraphics[width=\maxwidth{0.95\columnwidth}]{#1}}

\makeatletter
\def\maxwidth#1{\ifdim\Gin@nat@width>#1 #1\else\Gin@nat@width\fi}
\makeatother

\usepackage{soul}

\begin{document}

\maketitle

\section{Example string problems}

Following are a couple examples of string problems. They'll help you understand strings before you create a text adventure in the next section.

\begin{exampleproblem}
    \exampleproblemstatement
    Write a function,
    \vspace{3pt}
    \\
    \texttt{std::string get\_first\_name(std::string full\_name)}
    \vspace{3pt}
    \\
    that accepts a string \texttt{full\_name} representing a person's full name in the format ``last, first middle'', and returns the person's first name. For example, \texttt{get\_first\_name("Guyer, Alexander Dale")} should return \texttt{"Alexander"}.
    
    \exampleproblemsolution
    \begin{verbatim}
std::string get_first_name(std::string full_name) {
    int space_position = full_name.find(" ");
    std::string first_and_middle = full_name.substr(space_position + 1);
    space_position = first_and_middle.find(" ");
    std::string first_name = first_and_middle.substr(0, space_position);
    return first_name;
}
    \end{verbatim}

    \exampleproblemsolution
    \begin{verbatim}
std::string get_first_name(std::string full_name) {
    int space_position = full_name.find(" ");
    full_name.erase(0, space_position + 1);
    space_position = full_name.find(" ");
    full_name.erase(space_position);
    return full_name;
}
    \end{verbatim}
\end{exampleproblem}

\begin{exampleproblem}
    \exampleproblemstatement
    Write a function
    \vspace{3pt}
    \\
    \texttt{int count\_substr(std::string str, std::string substr) \{...\}}
    \vspace{3pt}
    \\
    that returns the number of times the string \texttt{substr} appears within the string \texttt{str}. Occurrences of \texttt{substr} within \texttt{str} are allowed to overlap (e.g., \texttt{count("ooo", "oo")} should return 2. The middle o is involved in both appearances: ``\underline{oo}o'' and ``o\underline{oo}'').

    \exampleproblemsolution
    \begin{verbatim}
int count_substr(std::string str, std::string substr) {
    int count = 0; // The number of occurrences found
    int cur_position = 0; // The position from which to search
    
    while (cur_position != std::string::npos) {
        int position = str.find(substr, cur_position);
        
        if (position == std::string::npos) {
            // Failed to find any more occurrences. Set cur_position =
            // position (= std::string::npos) to end the loop.
            cur_position = position;
        } else {
            // Found an occurrence.
            count++;
            // Set cur_position = position + 1 so that the next
            // iteration starts searching for the next occurrence.
            cur_position = position + 1;
        }
    }
    
    return count;   
}
    \end{verbatim}

    \exampleproblemsolution
    \begin{verbatim}
int count_substr(std::string str, std::string substr) {
    int count = 0; // The number of occurrences found
    int cur_position = 0; // The position from which to search
    
    // Advanced technique: using the assignment operation's
    // evaluation as part of the loop condition
    while ((cur_position = str.find(substr, cur_position)) != std::string::npos) {
        // Found an occurrence. Increment count
        count++;
        // Also increment cur_position so that the next loop
        // iteration starts searching for the next occurrence
        cur_position++;
    }
    
    return count;
}
    \end{verbatim}
\end{exampleproblem}

\section{Text-based game}

Write a text-based game to be played in the terminal. The content of your game is completely up to you (but it must be appropriate). The only requirements are:

\begin{enumerate}
    \item Your game must prompt the user for actions via \texttt{std::cout}.
    \item Each action must be read as a \texttt{string} via \texttt{getline(std::cin, some\_string\_variable)}.
    \item Each action must be checked for validity (e.g., if your program asks the user to enter ``north'' to go north or ``south'' to go south, then your program must verify that the user entered one of those two things).
    \item When a user enters an invalid action, your program should tell the user that their input was invalid, and it should prompt them again. You'll need some loops for this.
    \item The game's ``storyline'' must somehow depend on the user's chosen actions. You'll need some if statements for this.
    \item Your game must involve \underline{at least three} different prompts, each with their own set of valid actions (i.e., at least three different ``situations'' where the user has a choice to make). \underline{Note}: It's okay if certain actions cause the game to end before the user has made three choices---the user does not need to encounter all three prompts in a single playthrough.
\end{enumerate}

Here are some example runs of my text-based game:

\begin{terminaloutput}
./a.out
You wake up and find yourself in a desert. You feel weak and nauseous. There appears
to be an oasis in the distance. What do you do?

Type "walk toward oasis" to walk toward the oasis, or "walk into desert" to walk away
from the oasis, deeper into the desert:

drink water

That's not a valid action!

Type "walk toward oasis" to walk toward the oasis, or "walk into desert" to walk away
from the oasis, deeper into the desert:

walk toward oasis

You begin stumbling toward the oasis... After hours of walking, the oasis appears
just as far away as when you started. You're dehydrated, but you've stumbled across a
desert cactus. What do you do?

Type "walk toward oasis" to continue walking toward the oasis. Type
"drink cactus water" to drink water from the cactus.

drink cactus water

You drink the cactus water. It tastes acrid. Even worse, cactus water is apparently
not meant for human consumption. You feel weak... Your legs give out, and you pass
out in the hot desert sand.

Game over.

\end{terminaloutput}

\begin{terminaloutput}
./a.out
You wake up and find yourself in a desert. You feel weak and nauseous. There appears
to be an oasis in the distance. What do you do?

Type "walk toward oasis" to walk toward the oasis, or "walk into desert" to walk away
from the oasis, deeper into the desert:

walk into desert

You feel uneasy about the oasis, suspecting it to be a hallucination. You turn around
and walk deeper into the desert. After hours of walking, you come across a well-kept
road running from north to south, but you see the peak of a building emerging on the
eastern horizon. You feel dehydrated, and you're probably too exhausted to reach
it... What do you do?

Type "go east" to fight the odds and walk through the sand toward the building. Type
"go north" to walk north along the road. Type "go south" to walk south along the
road. Type "lay down" to lay down in the middle of the road.

lay down

You recognize that you're nearing your limits. You choose to lay down in the road to
conserve energy, hoping that help will arrive soon...

Your decision pays off. A caravan of merchants arrives from the north, apparently
on their way to sell their wares in a distant town to the south. They offer you water
and escort you to safety.

\end{terminaloutput}

\end{document}
